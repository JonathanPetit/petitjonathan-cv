%%%%%%%%%%%%%%%%%%%%%%%%%%%%%%%%%%%%%%%%%
% Twenty Seconds Resume/CV
% LaTeX Template
% Version 1.1 (8/1/17)
%
% This template has been downloaded from:
% http://www.LaTeXTemplates.com
%
% Original author:
% Carmine Spagnuolo (cspagnuolo@unisa.it) with major modifications by 
% Vel (vel@LaTeXTemplates.com)
%
% License:
% The MIT License (see included LICENSE file)
%
%%%%%%%%%%%%%%%%%%%%%%%%%%%%%%%%%%%%%%%%%

\documentclass[letterpaper]{twentysecondcv} 

\usepackage[utf8x]{inputenc}
\usepackage[T1]{fontenc}
\usepackage[french]{babel}

\usepackage{fontawesome}

\profilepic{Jonh.jpg} 

\cvname{Jonathan Petit}
\cvjobtitle{Ingénieur industriel/ \\ 1ère Master informatique}

\cvdate{16 Mars 1995} 
\cvaddress{Rue du blanc-ry, \newline 1340 Ottignies}
\cvnumberphone{+32 475268733}
\cvsite{JonathanPetit} 
\cvlinkedin{jonathan-petit16}
\cvmail{petit.jonathan16@gmail.com} 

\begin{document}

\aboutme{}

\skills{{Docker/1},{Git/3},{LATEX, SQL/5}, {GoLang, C/2}, {PhP, Javascript, HTML, CSS/3}, {Java, C\#, SWIFT/4}, {Python/5}}

\skillstext{}

\makeprofile 


\section{Formation}

\begin{twenty}
	\twentyitem{2008-2013}{CESS}{College Notre dame des Hayeffes, Mont-Saint-Guibert}{Etudes secondaires, options mathématiques et siences}
	\twentyitem{2013-2018}{Bachelier en sciences industrielles}{ECAM, Bruxelles}{Option génie éléctrique (informatique, électronique)}
	\twentyitem{2018-...}{Master en ingénieur industriel}{ECAM, Bruxelles}{Informatique}\\
	\twentyitem{2018}{BEPS}{Croix-rouge, Belgique}{Brevet Européen de Premiers Secours}\\
\end{twenty}


\section{Expérience Professionnelle}

\begin{twenty}
    \twentyitem{2017}{Cra-W}{Stage}{Développement d'une platforme WEB pour les centres pilotes wallons. Programmation backend à l'aide de Python (Framework Django) et de Javascript. Frontend à l'aide d'HTML et CSS.}\\
\end{twenty}


\section{Projets}

\begin{twentyshort} 
    \twentyitemshort{2017}{Participation à la compétition Eurobot avec l'ECAM. Aide dans la partie de détection des objets.}\\
	\twentyitemshort{2018}{Participation au développement d'une Application mobile en Swift pour permettre à des personnes handicapées de réaliser des exercices vocaux et visuels.} \\
	\twentyitemshort{Depuis 2018}{Projet personnel d'une application mobile de gestion de commandes en restaurant. Cette application est réalisé pour un événement scout. Elle permet d'envoyer les commandes en cuisine afin d'optimiser le rendement. Les serveurs reçoivent une notification lorsque la commande est prête. L'API est écrite GoLang et l'application mobile en Swift.}\\
	\twentyitemshort{Depuis 2019}{Projet de drône de livraison. Responsable de la partie GPS et du pathfinding.}\\
	
\end{twentyshort}


\section{Jobs étudiants}

\begin{twenty}
    \twentyitem{2013}{Cra-W, Unité de protection de l'eau et des sols.}{Job étudiant}{Multiples tests sur des échantillons de blés pour déterminer les caractéristiques des sols.}\\
\end{twenty}


\section{Langues}

\begin{twentyshort}
	\twentyitemshort{Français}{Langue maternelle}
	\twentyitemshort{Anglais}{Niveau intermédiaire. (Niveau B1)}
	\twentyitemshort{Néerlandais}{Niveau basique}\\
\end{twentyshort}


\section{Centre d'intérêts}

\begin{twentyshort} 
	\twentyitemshort{Football}{Joueur pendant 13 ans.}
	\twentyitemshort{Informatique}{Autodidacte}
	\twentyitemshort{Scout}{Chef depuis 6 ans fonction de trésorier. Organisateur de plusieurs événement de plus de 500 personnes.}
\end{twentyshort}

\end{document} 
